\chapter{集合、映射与函数}

关于集合、映射与函数,我们在高中时就已经十分熟悉了,学习本节,我们只需要简单理解其含义即可,另外我们会接触到有界性、邻域相关的概念,同时需要了解并记住一些常见的函数。在考试中,本节内容占比不大,因此本节对于高中已学的内容不做过多叙述,只需抓住一些重点即可。

\section{集合}
\begin{enumerate}
	\item 集合: 具有某种特定性质的对象的全体,组成这个集合的个别对象称为该集合的元素。
	\item 区间:是指介于某两个实数之间的全体实数, 这两个实数叫做区间的端点。
	\item \textbf{邻域}: 设$a$与$\delta$是两个实数,且$\delta$>0。数集$\{x \mid |x-a|<\delta\}$称为点$a$的$\delta$邻域,点$a$叫做这邻域的中心,$\delta$叫做这邻域的半径。记作$U_\delta(a)=\{x\mid a-\delta<x<a+\delta\}$。

	点$a$的去心邻域,记作$\mathring{U}(a)=\{x \mid 0<|x-a|< \delta \}$。
\end{enumerate}

\section{有界性}
\begin{enumerate}
	\item 上有界:$\exists L \in \mathbb{R}$,使得$\forall x \in A$,都有$x\leq L$;

	\item 下有界:$\exists l \in \mathbb{R}$,使得$\forall x \in A$,都有$x \geq l$;
	
	\item 集合$A$有界:$A$既有上界又有下界。
	
	\item \textbf{确界}:上确界为最小上界,下确界为最小下界(可类比函数的最大最小值来理解)
	
	设$A$为非空实数集,若$\exists s \in \mathbb{R}$,满足:
	\begin{enumerate}
		\item $\forall x \in A$,$x \geqslant $($\leqslant$)$s$;
		\item $\forall \varepsilon > 0$,$\exists x_0 \in A$,使$x_0 >(<) s -(+) \varepsilon$。
	\end{enumerate}
	则$s$为A的上确界(下确界),记作$\sup A$($\inf A$)。
	\begin{remark}
		此类定义语言对非数学专业的学生要求不高(包括后面的$\varepsilon$-$\delta$语言,只需简单理解即可,理解时只需将$\epsilon$看成一个很小很小的数。
	\end{remark}
\end{enumerate}

\section{映射和函数}
早在初中我们便已学过映射和函数的相关概念,具体内容可参考高数课本。
\subsection{几种特殊函数}
\begin{enumerate}
	\item 取整函数,$y=[x]$,$[x]$表示不超过$x$的最大整数。
	\item 符号函数,
	\begin{eqnarray}
		y={\rm sgn} x=
		\begin{cases}
			1,  & x>0 \\
			0,  & x=0 \\
			-1, & x<0
		\end{cases}
	\end{eqnarray}
	\item 狄利克雷函数,
	\begin{eqnarray}
		y=D(x)=
		\begin{cases}
			1, & x\in \mathbb{Q} \\
			0, & x\in \mathbb{R} \backslash \mathbb{Q}
		\end{cases}
	\end{eqnarray}
	\item 取最值函数,
	如$y={\rm max}\{f(x),g(x)\}$,$y={\rm min}\{f(x),g(x)\}$,
	如在题目中遇到此类取最值函数,只需画出函数图像,便可简单转为分段函数来处理。
\end{enumerate}

\subsection{反函数}

对于反函数,我们只需掌握反函数与原函数的函数图像关于直线y=x对成即可。常见的反函数有反三角函数,需要我们掌握。

$y=\sin x$对应的反函数为$y=\arcsin x$,$y=\cos x $对应的反函数为$y=\arccos x$,$y=\tan x $对应的反函数为$y=\arctan x$.

在遇到与反函数有关的问题时,可从两者图像关于$y=x$对称入手,在需要用到与反函数有关的性质时,也可通过画原函数关于$y=x$对称的函数图像来解决。

总的来说,关于反函数,我们只需要理解反函数是如何得到的,以及会用三种常见的反三角函数(后面关于反三角函数的导数也只需要记住这三种常见的即可),另外记住反函数与原函数的函数图像是关于$y=x$对称的。
\begin{example}
	设$f: \mathbb{R} \to \mathbb{R}$ 单调递增,$f^{-1}$是其反函数,$x_1$是方程$f(x)+x=a$的根,$x_2$是方程$f^{-1}(x)+x=a$的根,求$x_1$+$x_2$的值。
	\begin{solution}
		由$x_1$是方程$f(x)+x=a$的根,可知$f(x_1)+f^{-1}[f(x)]=a$因为$x=f^{-1}[f(x)]$。
		
		所以$f(x_1)$是方程$f^{-1}(x)+x=a$的根,又$f$单增,则$f^{-1}(x)+x$单增,可以得到方程$f^{-1}(x)+x=a$只有一个根,所以$f(x_1)=x_2$,所以$x_1+x_2=x_1+f(x_1)=a$。
	\end{solution}
\end{example}

\section{一些补充}
\subsection{双曲函数}

双曲正弦: ${\rm sh} x=\frac{{\rm e}^x-{\rm e}^{-x}}{2}$

双曲余弦:${\rm ch} x=\frac{{\rm e}^x+{\rm e}^{-x}}{2}$

双曲正切:${\rm th} x=\frac{{\rm e}^x-{\rm e}^{-x}}{{\rm e}^x+{\rm e}^{-x}}$
\begin{remark}
	关于双曲函数,我们需要知道它与三角函数是十分类似的,具体一些性质可参考课本(类比三角函数去记忆即可),在后面不定积分的计算以及其它问题中,我们可能会需要用到双曲函数。因此需要有一个关于这个函数的印象。
\end{remark}

\subsection{极坐标与参数方程}

极坐标与参数方程是高中选修4-4的内容,如果高中未学过选修4-4,可参照高数课本的附录1.

\subsection{积化和差和和差化积公式}

关于积化和差和和差化积公式,在后续的解题中会偶尔用到(尤其是在高数下的傅里叶变换求积分的过程中),积化和差和和差化积公式的本质是和角公式和差角公式,可参考高数课本附录。

\subsection{一些常见曲线及其图像}

常见的曲线有摆线、星形线。心形线、双纽线、玫瑰线\mn{在高数下的二重积分部分应用比较多,详情可参考高数课本附录。}。