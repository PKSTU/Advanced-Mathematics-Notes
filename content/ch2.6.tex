\chapter{函数性态的研究}\label{ch:6}

本节的内容是对前五节内容的综合和提升,在学习过程中,同学们要学以致用,温故知新,具体需要大家:

\begin{enumerate}
	\item \textbf{熟练掌握}函数的极值,单调区间,最值的求法;
	\item \textbf{深刻理解}函数的凹凸性,\textbf{熟练掌握}函数的凹凸区间和拐点的求法;
	\item \textbf{掌握}函数各种渐近线的求法,会大致作出函数的图像。
\end{enumerate}

\section{复习背景}
\subsection{函数的极值}
\begin{definition}
	设有函数$f:I\to\mathbb{R}$,若$\exists\delta>0$,使得$\forall x\in U(x_0,\delta)\subseteq I$,恒有$f(x)\geq f(x_0)(\leq f(x_0))$,则称$f$在$x_0$取得极小(大)值$f(x_0)$。$f$的极小值与极大值统称为$f$的极值,使$f$取得极值点的点$x_0$称为$f$的极值点。
\end{definition}

\begin{remark}
	极值点指的是取得极值时\textbf{自变量的值},并不是极值点的坐标。
\end{remark}

\subsection{Lagrange定理}
\begin{theorem}
	若函数$f:[a,b]\to\mathbb{R}$满足条件:(1)$f$在$[a,b]$连续;\quad(2)$f$在$(a,b)$内可导;则至少存在一点$\xi\in (a,b)$,使
	\begin{equation}
		f(b)-f(a)=f'(\xi)(b-a)\label{eq:6.1}
	\end{equation}
\end{theorem}

\subsection{Taylor定理}
\begin{theorem}
	设函数$f$在区间$I$上$n+1$阶可导,$x_0\in I$,则对任何$x\in I$,在$x$与$x_0$之间至少存在一点$\xi$,使得
	\begin{equation}
		f(x)=\sum^n_{k=0}\frac{f^{(k)}(x_0)}{k!}(x-x_0)^k+\frac{f^{(n+1)}(\xi)}{(n+1)!}(x-x_0)^{n+1}\label{eq:6.2}
	\end{equation}
\end{theorem}
\section{知识点初始——函数的单调性}
设$f:I\to\mathbb{R}$在$I$上连续,在$I$内可导,则以下命题成立:
\begin{enumerate}
	\item $f$在$I$上单调增(减)的充要条件是在$I$内$f'\geq 0$($f'\leq 0$);
	\item 若在$I$内$f'>0$($f'<0$),则$f$在$I$上严格单调增(减)\mn{若$f$在$I$上严格单调增(减),$f'$在$I$内不一定处处为正(负)。}。
\end{enumerate}
\begin{remark}
	为了判定给定函数$f$的单调区间,应当先求出$f'=0$的根以及不可导的点,再分区间讨论函数的单调性。
\end{remark}
\section{知识点初始——函数的极值}
\subsection{函数的驻点}
将使$f'(x)=0$的点称为$f$的驻点,可导函数的极值点必定是驻点。
\subsection{函数极值点的第一条件}
设$f$在$x_0$的某邻域$U(x_0)$内可导,并且$f'(x_0)=0$,
\begin{enumerate}
	\item 若$x<x_0$时$f'(x)\geq 0$,$x>x_0$时$f'(x)\leq 0$,则$f$在$x_0$处取最大值;
	\item 若$x<x_0$时$f'(x)\leq 0$,$x>x_0$时$f'(x)\geq 0$,则$f$在$x_0$处取最小值;
	\item 若$f'(x)$在$x_0$的左右两侧符号不变,则$f$在$x_0$处不取极值。
\end{enumerate}
\begin{remark}
	在讨论极值点时,\textbf{一定不能忽略不可导的点};应当\textbf{利用极值点的定义}来判断不可导的点,这一点在考试中经常作为考点出现。
\end{remark}
\subsection{函数极值点的第二条件}
设$f$在$x_0$处\textbf{二阶可导},并且$f'(x_0)=0,f''(x_0)\neq 0$,则当$f''(x_0)>0(<0)$时,$f$在$x_0$处取极小(大)值。
\subsection{函数极值点的第三条件}
设$f$在$x_0$处\textbf{$n(n\geq 2)$阶可导},并且$f'(x_0)=f''(x_0)=\cdots =f^{(n-1)}(x_0)=0,f^{(n)}(x_0)\neq 0$,
\begin{enumerate}
	\item 当$n$为偶数时,$x_0$必为极值点,若$f^{(n)}(x_0)>0$,则$x_0$为极小值点;若$f^{(n)}(x_0)<0$,则$x_0$为极大值点;
	\item 当$n$为奇数时,$x_0$不是极值点。
\end{enumerate}
\section{知识点初始——函数的最值}
\subsection{函数最值的求法}
若$f\in C[a,b]$,则函数必定在区间上有最值,最值可能出现在:
\begin{enumerate}
	\item 区间内的极值点(可能是驻点或不可导的点);
	\item 区间的端点。
\end{enumerate}

求出极值点,再将其与端点值比较,即可得到最值。
\subsection{函数最值的性质}
区间$I$上的最大值$M$(最小值$m$)有如下性质\mn{该性质也是证明不等式的常用方法之一。}:
\[f(x)\leq M(f(x)\geq m),x\in I\]
\section{知识点初始——函数图像的凹凸性与拐点}
不同教材对凹凸性的定义不同,此处特别说明是从图像的几何特性上定义的。
\subsection{函数图像的凹凸性}
\subsubsection{凹凸性的定义}
\begin{definition}
	设函数$y=f(x)$在区间$I$上连续,若对$I$中任意两点,曲线上对应的弧段始终位于两点连线构成的弦的下(上)方,则称函数$f(x)$在区间$I$中的图像是凹(凸)的。
\end{definition}
\subsubsection{凹凸性的性质}
若$f(x)$在$I$上是凹的,则对$\forall x_1,x_2\in I,\lambda\in (0,1)$,有
\[f(\lambda x_1+(1-\lambda x_2))<\lambda f(x_1)+(1-\lambda)f(x_2)\]

即自变量线性组合的函数值总小于函数的对应线性组合。

\textbf{特别地,当$\lambda =\frac{1}{2}$时},有
\[f(\frac{x_1+x_2}{2})<\frac{f(x_1)+f(x_2)}{2}\]

函数图像为凸的时候有对应性质\mn{函数图像凹凸性的性质在不等式的证明当中经常用到,特别是系数为$\frac{1}{2}$的形式。}。
\subsection{函数的拐点}
\subsubsection{拐点的定义}
\begin{definition}
	连续曲线上凹的图像与凸的图像的转变点称为此曲线的拐点,拐点处二阶导数为0。
\end{definition}
\begin{remark}
	拐点是用\textbf{平面有序数组}表示的,是真正意义上的点,注意与极值点区分。
\end{remark}
\subsubsection{拐点的第一充分条件}
若$f(x)$二阶可导,且$f''(x_0)$为$f''$的变号零点,则$(x_0,f(x_0))$为$f$的拐点。
\subsubsection{拐点的第二充分条件}
若$f(x)$在$x_0$的去心邻域内可导,且$f''$在$x_0$左右变号,则$(x_0,f(x_0))$为$f$的拐点\mn{该条件说明二阶导数不连续的点也有可能是函数的拐点。}。
\subsubsection{拐点的第三充分条件}
若$f(x)$在$U(x_0,\delta)$内三阶可导,且$f''(x_0)=0,f'''(x_0)\neq 0$,则$(x_0,f(x_0))$为$f$的拐点。
\subsubsection{拐点的第四充分条件}
设$f$在$x_0$处\textbf{$n(n\geq 3)$阶可导},并且$f'''(x_0)=f^{(4)}(x_0)=\cdots =f^{(n-1)}(x_0)=0,f^{(n)}(x_0)\neq 0$,
\begin{enumerate}
	\item 当$n$为奇数时,$(x_0,f(x_0))$是拐点;
	\item 当$n$为偶数时,$(x_0,f(x_0))$不是拐点。
\end{enumerate}
\section{函数的渐近线与函数作图}
\subsection{函数的渐近线}
\subsubsection{水平渐近线}
$\lim_{x\to\infty}f(x)=A\xrightarrow{} y=A$为水平渐近线(单侧极限成立即可)。
\subsubsection{铅直渐近线}
$\lim_{x\to a}f(x)=\infty\xrightarrow{}x=a$为铅直渐近线(可以分别求左右极限,一般关注函数的间断点)。
\subsubsection{斜渐近线}
$\lim_{x\to\infty}\frac{f(x)}{x}=k,\quad \lim_{x\to\infty}[f(x)-kx]=b\xrightarrow{}y=kx+b$为斜渐近线(单侧极限成立即可)。
\subsection{函数作图的步骤}
(1)求定义域和间断点;

(2)考察奇偶性,周期性,有界性;

(3)求出一阶导数,二阶导数的零点和不可导的点;

(4)判断单调区间,极值点,凹凸区间,拐点;

(5)求渐近线;

(6)结合特殊位置函数值,描点画图,画出渐近线。
\section{序轴标根法刻画多项式函数的图像}
在遇到高阶多项式以最简因式形式给出时,可以利用该方法大致刻画多项式的形态:

(1)将自变量的系数全部整理为正数;

(2)将函数的零点按数轴位置标在$x$轴上;

(3)若多项式系数为正,自左上方开始由正无穷向下,遇到根的对应因式为奇数次方就穿过,遇到根的对应因式为偶数次方就与它相切,即“奇穿偶切”,经过每一个零点;若多项式系数为负,则图像关于$x$轴对称即可。
\begin{remark}
	(1)读者可以自行探究该方法的原理,在遇到变形时(如加入绝对值)可以灵活应对;
	(2)该方法多用于解决高阶多项式的不等式问题,高阶多项式的极值点,拐点问题。
\end{remark}
\section{证明不等式的常见方法}
(1)利用导数的定义;

(2)利用微分中值定理;

(3)利用函数的单调性;

(4)利用Taylor公式;

(5)利用函数的极值和最值;

(6)利用函数的凹凸性。

\section{习题}
\begin{problem}
	已知$f(x)$在$x=0$某邻域内连续,$\lim_{x\to 0}\frac{f(x)}{1-\cos{x}}=2$,则在$x=0$处$f(x)$\xparen
	\begin{xchoices}[showanswer=true]
		\item 不可导
		\item 可导且$f'(0)\neq 0$
		\item 取得最大值
		\item* 取得最小值
	\end{xchoices}
	\vspace{0.5em}
	\begin{solution}
		由于$\lim_{x\to 0}\frac{f(x)}{1-\cos{x}}=2>0$,由极限的保号性,存在$x=0$的去心邻域,在此去心邻域内$\frac{f(x)}{1-\cos{x}}>0$,又$1-\cos{x}>0$,则在此去心邻域内$f(x)>0$。再由$\lim_{x\to 0}\frac{f(x)}{1-\cos{x}}=2$知,$\lim_{x\to 0}f(x)=0$,又$f(x)$在$x=0$连续,从而$f(0)=0$,则存在$x=0$的邻域,在此邻域内$f(x)\geq f(0)$,则$f(x)$在$x=0$处取极小值,故选(D)
	\end{solution}
\end{problem}
\begin{remark}
	本题\textbf{不能对已知极限用L'Hospital法则},因为原题只假定$f(x)$连续。
\end{remark}

\begin{problem}
	证明:$\tan{x}=1-x$在$(0,1)$内有唯一实根。
	\begin{proof}
		令$f(x)=\tan{x}-1+x$,显然$f(x)$在$[0,1]$上连续,且$f(0)=-1<0,f(1)=\tan{1}>0$,由零点定理知,$f(x)$在$(0,1)$内至少有一个零点,即方程$\tan{x}=1-x$在$(0,1)$内至少有一个实根;又
        \[f'(x)=\sec^2{x}+1>0,\quad x\in [0,1]\]

        则$f(x)$在$(0,1)$上严格单调增,$f(x)$在$(0,1)$内最多一个零点,即方程$\tan{x}=1-x$在$(0,1)$内最多有一个实根;综上方程$\tan{x}=1-x$在$(0,1)$内有且只有一个实根。
	\end{proof}
\end{problem}
\begin{problem}
	设$f''(x)<0,f(0)=0$,证明:对任何$x_1>0,x_2>0$,有$f(x_1+x_2)<f(x_1)+f(x_2)$
	\begin{solution}
		令\[G(x)=f(x_1+x)-f(x)-f(x_1)\]
        \[G'(x)=f'(x_1+x)-f'(x)=f''(\xi)x_1<0\quad(x<\xi <x+x_1)\]

        则$G(x)$单调减,从而有$G(x_2)<G(0)$,即
        \[f(x_1+x_2)-f(x_2)-f(x_1)<f(x_1)-f(0)-f(x_1)=0\]
        \[\Rightarrow f(x_1+x_2)<f(x_1)+f(x_2)\]
	\end{solution}
\end{problem}

\begin{problem}
	设$p,q>0$,且$\frac{1}{p}+\frac{1}{q}=1$,又设$a>0,b>0$,求证:$ab\leq \frac{1}{p}a^p+\frac{1}{q}b^q$。
	\begin{proof}
		对不等式两端同时取对数
        \[\ln{(ab)}\leq \ln{(\frac{1}{p}a^p+\frac{1}{q}b^q)}\]

        令$f(x)=\ln{x},\quad x\in (0,+\infty)$,则$f''(x)=-\frac{1}{x^2},\quad x\in (0,+\infty)$

        $f(x)=\ln{x}$为$(0,+\infty)$上的凹函数,从而有
        \[f\qty(\frac{1}{p}a^p+\frac{1}{q}b^q)\geq \frac{1}{p}f(a^p)+\frac{1}{q}f(b^q)\]
        \[\ln{(\frac{1}{p}a^p+\frac{1}{q}b^q)}\geq \frac{1}{p}a^p+\frac{1}{q}b^q=\ln{a}+\ln{b}=\ln{(ab)}\]
        \[ab\leq \frac{1}{p}a^p+\frac{1}{q}b^q\]
	\end{proof}
\end{problem}

\begin{problem}
	证明不等式
    \[\cos{\sqrt{2}x}<-x^2+\sqrt{1+x^4}\quad x\in (0,\frac{\sqrt{2}}{4}\pi)\]
	\begin{proof}
		令
        \[f(x)=\sqrt{1+x^4}-x^2-\cos{\sqrt{2}x}\]
        \[f'(x)=2x(\frac{x^2}{\sqrt{1+x^4}}-1)+\sqrt{2}\sin{\sqrt{2}x}\]

        由Taylor展开式,易得
        \[\sin{x}>x-\frac{1}{3!}x^3(x>0)\]

        所以
        \[f'(x)>2x(\frac{x^2}{\sqrt{1+x^4}}-1)+2x-\frac{2x^3}{3}=\frac{2x^3}{\sqrt{1+x^4}}-\frac{2x^3}{3}>0\]

        因此,当$x\in (0,\frac{\sqrt{2}}{4}\pi)$时,$f(x)$单调增加,又$f(0)=0$,所以$f(x)>0$。
	\end{proof}
\end{problem}
\newpage
\begin{problem}[思考题]
	证明\textbf{赫尔德不等式}:设实数$\alpha,\beta$满足$\frac{1}{\alpha}+\frac{1}{\beta}=1$,且$a_i,b_i(i=1,\cdots,n)$为非负实数,则当$\alpha>1$时
    \[\sum^n_{i=1}a_i b_i\leq (\sum^n_{i=1}a^\alpha_i)^{\frac{1}{\alpha}}(\sum^n_{i=1}b^\beta_i)^{\frac{1}{\beta}}\]

    当$\alpha<1$时
    \[\sum^n_{i=1}a_i b_i\geq (\sum^n_{i=1}a^\alpha_i)^{\frac{1}{\alpha}}(\sum^n_{i=1}b^\beta_i)^{\frac{1}{\beta}}\]
\end{problem}

\begin{proof}
	记$p_i=b_i^\beta$,由$\frac{1}{\alpha}+\frac{1}{\beta}=1$易得,当$\alpha<1$时,原不等式等价于
	\[\sum^n_{i=1}a_i p_i^{1-\frac{1}{\alpha}}\leq (\sum^n_{i=1}a^\alpha_i)^{\frac{1}{\alpha}}(\sum^n_{i=1}p_i)^{1-\frac{1}{\alpha}}\]

	等价于
	\[\sum^n_{i=1}(\frac{p_i}{\sum^n_{i=1}p_i})\alpha_i p_i^{-\frac{1}{\alpha}}\leq (\sum^n_{i=1}(\frac{a_i^\alpha}{\sum^n_{i=1}p_i}))^{\frac{1}{\alpha}}\]

	记$x_i=a_i p_i^{-\frac{1}{\alpha}},\lambda_i=\frac{p_i}{\sum^n_{i=1}p_i}$,则上面不等式化为
	\[(\sum^n_{i=1}\lambda_i x_i)^\alpha\leq \sum^n_{i=1}\lambda_i x_i^\alpha\]

	且$\sum^n_{i=1}\lambda_i=1$,故只需讨论函数$f(x)=x^\alpha(\alpha>1)$的凸性。

	令$f(x)=x^\alpha,\quad x\geq 0$,则$f''(x)=\alpha(\alpha-1)x^\alpha$,当$\alpha>1$时,$f''(x)>0$,$f(x)$是下凸函数。

	对任意一组非负实数$p_i(i=1,\cdots,n)$,记$\lambda_i=\frac{p_i}{\sum^n_{i=1}p_i}$,则$\sum^n_{i=1}\lambda_i=1$,对任意$x_i\geq 0$,利用凸函数的Jensen不等式,得
	\[(\sum^n_{i=1}\lambda_i x_i)^\alpha\leq \sum^n_{i=1}\lambda_i x_i^\alpha\]
	
	即
	\[\qty(\sum^n_{i=1}\frac{p_i x_i}{\sum^n_{i=1}p_i})^\alpha\leq \sum^n_{i=1}\frac{p_i x_i}{\sum^n_{i=1}p_i}\]
	
	即
	\[\sum^n_{i=1}p_i x_i\leq \qty(\sum^n_{i=1}a^\alpha_i)^{\frac{1}{\alpha}}\qty(\sum^n_{i=1}p_i)^{1-\frac{1}{\alpha}}\]

	取$a_i=p_i^{\frac{1}{\alpha}},b_i=p_i^{\frac{1}{\beta}}$,则有
	\[\sum^n_{i=1}a_i b_i\leq \qty(\sum^n_{i=1}a^\alpha_i)^{\frac{1}{\alpha}}\qty(\sum^n_{i=1}b^\beta_i)^{\frac{1}{\beta}}\]
	
	同理可证$\alpha<1$的情况。
\end{proof}

\begin{remark}
	(1)当$\alpha=\beta=2$时,赫尔德不等式即为常见的\textbf{柯西不等式};

	(2)凸函数的\textbf{Jensen不等式}:

	设$f(x)\in C(I)$,则$f(x)$是$I$上下凸函数的充要条件为:$\forall x_i\in I,\forall \lambda\in(0,1)(i=1,\cdots,n)$,且$\sum^n_{i=1}\lambda_i=1$,恒有
	\[f\qty(\sum^n_{i=1}\lambda_i x_i)\leq \sum^n_{i=1}\lambda_i f(x_i)\]
\end{remark}