\chapter{无穷小量与无穷大量}

本节课主要介绍无穷小量和无穷大量的概念。无穷小量在求函数极限的过程中有重要作用,无穷小量也广泛应用于函数(数列)极限阶的估计。
	
	本节课的主要内容包括:
	\begin{enumerate}
		\item 无穷小量的概念与性质;
		\item 无穷小在求函数极限过程中的应用;
		\item 无穷大量。
	\end{enumerate}

	\section{无穷小量的概念与性质}
	
	\begin{definition}[\textbf{无穷小量}]
		当$x\to x_0\left(x\to \infty\right)$时,以零为极限的函数$\alpha(x)$称为当$x\to x_0\left(x\to \infty\right)$时的\textbf{无穷小量},简称为\textbf{无穷小}.
	\end{definition}

	例如,当$x\to 0$时,$x^2,\sin x,\tan x$都是无穷小;当$x\to \infty \text{时},\frac{1}{x},\frac{\sin x}{x}$都是无穷小.
	
	同时,以零为极限的数列也是无穷小($n\to \infty$).
	
	$\star$注意,无穷小量是以零为极限的\textbf{变量},而不是一个绝对值很小的常数.
	
	下面介绍三个无穷小量的定理.
	\begin{theorem}
		$\lim\limits_{}f(x)=a \text{当且仅当}f(x)=a+\alpha(x)$,其中$\alpha(x)$是无穷小.
	\end{theorem}

	\begin{theorem}
		在自变量有相同变化趋势的条件下。有
		\begin{enumerate}
			\item 有限个无穷小量的代数和是无穷小量;
			\item 有限个无穷小量的乘积是无穷小量.
		\end{enumerate}
	\end{theorem}

	\begin{theorem}
		\label{the3}
		设$\alpha(x)$是当$x\to x_0$时的无穷小,$f$是在$x_0$处局部有界的函数,则$\alpha (x)f(x)$是当$x\to x_0$时的无穷小.
	\end{theorem}
	
	定理\ref{the3}也就是说:\textbf{无穷小与有界函数的积还是无穷小}.
	
	\begin{example}
		求极限$\lim\limits_{x\to 0}x\sin\frac{1}{x}$.
	\end{example}

	\begin{solution}
		由于当$x\to 0$时,$x$是无穷小,并且当$x\ne 0$时,$\left|\sin\frac{1}{x}\right| \le 1$,故$\sin\frac{1}{x}$在$x=0$的任一去心邻域内是有界函数,所以由定理\ref{the3},
		\[
		\lim_{x\to 0}x\sin\frac{1}{x}=0.
		\]
	\end{solution}

	\section{无穷小的比较}
	首先,我们观察以下三个极限:
	\begin{enumerate}
		\item $\lim\limits_{x\to 0}\frac{x^2}{x}=0$;
		\item $\lim\limits_{x\to 0}\frac{2x}{x}=2$;
		\item $\lim\limits_{x\to 0}\frac{\sin{x}}{x}=1$.
	\end{enumerate}

	它们的分子分母都是无穷小量,但是比值的极限却各不相同.我们会想为什么不同无穷小比值的极限存在差异?是否可以通过某种方法来比较两个无穷小量“谁更小”.
	
	这里,我们引入无穷小阶的概念来解决这个问题.
	\begin{definition}
		设$\alpha (x)\text{与} \beta (x)$是自变量$x$有相同变化趋势的无穷小,且$\beta (x)\ne 0.$
		\begin{enumerate}
			\item 
			若$\lim\limits_{}\frac{\alpha(x)}{\beta (x)}=0$,则称$\alpha(x)$是$\beta(x)$的\textbf{高阶无穷小},或称$\beta(x)$是$\alpha(x)$的\textbf{低阶无穷小},记作$\alpha(x)=o(\beta(x))$.特别,一个无穷小$\alpha(x)$可记作$o(1)$.
			\item 
			若$\lim\limits_{}\frac{\alpha(x)}{\beta (x)}=c$,且$c\ne 0$为常数,则称$\alpha(x)$与$\beta(x)$是\textbf{同阶无穷小}.
			\item 
			若$\lim\limits_{}\frac{\alpha(x)}{\beta (x)}=1$,则称$\alpha(x)$与$\beta(x)$是\textbf{等价无穷小},记作$\alpha (x)\sim\beta(x)$.
			\item 
			若$\lim\limits_{}\frac{\alpha(x)}{(\beta (x))^k}=c$,其中$c\ne 0$为常数,$k>0$,则称$\alpha(x)$是\textbf{关于$\beta(x)$的$k$阶无穷小}.特别地,若取$\beta(x)=x-x_0$,若$\lim\limits_{x\to x_0}\frac{\alpha(x)}{(x-x_0)^k}=c$,则称$\alpha (x)$是当$x\to x_0$时的\textbf{$k$阶无穷小}.
		\end{enumerate}
	\end{definition}
	例如,当$x\to 0$时,
	
	$x^3+2x^2$与$2x^2$是等价无穷小,即$x^3+2x^2 \sim 2x^2$;
	
	$\sin{x}$与$x$是等价无穷小,即$\sin{x}\sim x$.
	
	\begin{theorem}
		设$\alpha(x)$与$\beta(x)$是在自变量同一变化趋势下的无穷小,且$\alpha(x)\sim \beta(x)$,则$\alpha(x)=\beta(x)+o(\beta(x))$或$\beta(x)=\alpha(x)+o(\alpha(x)).$
	\end{theorem}
	下面介绍几个常用的等价无穷小:
	
	当$x\to 0$时,
	\begin{enumerate}
		\item $\sin{x}\sim x$;
		\item $\tan{x}\sim x$;
		\item $\arctan{x}\sim x$;
		\item $1-\cos{x}\sim \frac{x^2}{2}$;
		\item $e^x-1\sim x$;
		\item $\ln{(1+x)}\sim x$;
		\item $(1+x)^{\alpha }-1\sim \alpha x$.
	\end{enumerate}
	\textbf{$\star$以上式子非常重要,请同学们务必掌握.}

	\section{无穷小的等价替换}
	\begin{theorem}[无穷小等价代换定理]
		设$\alpha(x)$与$\beta(x)$,$\tilde{\alpha}(x)$与$\tilde{\beta}(x)$都是在自变量同一变化趋势下的无穷小.若$\alpha(x)\sim \beta(x)$,$\tilde{\alpha}(x)\sim \tilde{\beta}(x)$,并且$\lim\limits_{}\frac{\tilde{\alpha}(x)}{\tilde{\beta}(x)}$存在,则$\lim\limits_{}\frac{\alpha(x)}{\beta(x)}$也存在,并且
			\[
			\lim\frac{\alpha(x)}{\beta(x)}=\lim\frac{\tilde{\alpha}(x)}{\tilde{\beta}(x)}.
			\]
	\end{theorem}

	这是一个非常重要的定理,经常用于极限的计算.它可以将$\frac{0}{0}$型不定式极限的分子分母用更简单的等价无穷小代换,便于计算.下面给出几个例子.
	
	\begin{example}
		求$\lim\limits_{x\to 0}\frac{\sqrt[3]{1+2x^2}-1}{\arcsin\frac{x}{2}\arctan\frac{x}{3}}$.
	\end{example}
	\begin{solution}
		由于当$x\to 0$时,
		\[
		(1+2x^2)^{\frac{1}{3} }-1\sim \frac{2}{3}x^2,
		\]
		\[
		\arcsin\frac{x}{2}\sim \frac{x}{2},
		\]
		\[
		\arctan\frac{x}{3}\sim \frac{x}{3},
		\]
		所以,
		\[
		\lim_{x\to 0}\frac{\sqrt[3]{1+2x^2}-1}{\arcsin\frac{x}{2}\arctan\frac{x}{3}}=\lim_{x\to 0}\frac{\frac{2}{3}x^2}{\frac{x}{2}\frac{x}{3}}=4.
		\]
	\end{solution}

	\begin{example}
		求$\lim\limits_{x\to 0}\frac{(1+x)^{\sqrt{2} }-1 }{x} .$
	\end{example}
	\begin{solution}
		\[
		\lim\limits_{x\to 0}\frac{(1+x)^{\sqrt{2} }-1 }{x}=\lim_{x\to 0}\frac{\sqrt{2}x}{x}=\sqrt{2}.
		\]
	\end{solution}

	\section{无穷大量}
	我们已经介绍过无穷小量,无穷大量与无穷小量相似,只是无穷大量的变化状态正好相反.
	首先,我们来定义无穷大量.
	
	\begin{definition}[\textbf{无穷大量}]
		设$f:\overset{\circ}{U}(x_0)\to \mathbb{R}$是一个函数,若$\lim\limits_{x\to x_0}f(x)=\infty$,则称函数$f(x)$是当$x\to x_0$时的\textbf{无穷大量}.
	\end{definition}
	\begin{itemize}
		\item 若$\lim\limits_{x\to x_0}f(x)=+\infty$,则称$f(x)$是当$x\to x_0$时的\textbf{正无穷大}.
		\item 若$\lim\limits_{x\to x_0}f(x)=-\infty$,则称$f(x)$是当$x\to x_0$时的\textbf{负无穷大}.
	\end{itemize}
	
	无穷大量和无穷小量之间有如下关系:
	\begin{theorem}
		\begin{enumerate}
			\item 若$f(x)$是无穷小量,且$f(x)\ne 0$,则$\frac{1}{f(x)}$是无穷大量;
			\item 若$f(x)$是无穷大量,则$\frac{1}{f(x)}$是无穷小量.
		\end{enumerate}
	\end{theorem}

	与无穷小量相似,无穷大量有以下运算性质:
	\begin{theorem}
	\begin{enumerate}
		\item 有限个无穷大量的乘积是无穷大量.
		\item 无穷大量与有界量之和是无穷大量.
	\end{enumerate}
	\end{theorem}