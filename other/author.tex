\chapter*{编写人员简介}
\addcontentsline{toc}{chapter}{编写人员简介}

\vspace{-3cm}

(按首拼字母排序)

\vspace{1em}

{\zihao{-2}\color{lbdeepblue}司秉钤}\quad 数试2101班学生,彭康学导团志愿者,在本份讲义中负责第一部分第4、5、6章编写。欢迎同学们来彭康学导团一起学习交流。同学们如有学习上的问题或者对本讲义有相关建议,请联系邮箱\url{2360804879@qq.com},希望和同学们一起学习进步。

\vspace{1em}

{\zihao{-2}\color{lbdeepblue}袁方}\quad 信息2105班学生,彭康学导团志愿者部成员,在本此讲义中负责第一部分第1、2、3章的编写。本人才疏学浅,所写不周之处还需要大家多多指正。最后希望大家认真学习,勤于思考,培养对数学学习的兴趣。

\vspace{1em}

{\zihao{-2}\color{lbdeepblue}许祺}\quad 金禾2101班学生,彭康学导团热心志愿者,在本份讲义中负责第二部分第7、9、11章的编写。啥也不会,需要学习。对本讲义提出意见/勘误/合作请联系邮箱\url{2977038022@qq.com}。

\vspace{1em}

{\zihao{-2}\color{lbdeepblue}赵程文轩}\quad 新能源2101班学生,2022年加入彭康学导团志愿者部,在本份讲义中负责第二部分第8、10、12章的编写。乐于探讨数学问题,欢迎同学们在闲暇之余来到彭康学导团讨论交流。第一次接触资料编写,有不当之处,欢迎大家批评指正。希望与同学们共勉,在数学学习的道路上共同进步。

\vspace{1em}

{\zihao{-2}\color{lbdeepblue}张恺}\quad 核工A002班学生,彭康学导团志愿者部部长,乐于分享,偏爱\LaTeX 排版,在本份讲义中负责全书的排版。自2021年加入彭康学导团大家庭以来,负责多份资料的编写和排版任务,包括高数等课程的真题集、《大学物理笔记(上、下)》、《流体力学·复习要点》等。

\vspace{1em}

{\zihao{-2}\color{lbdeepblue}赵钦}\quad 自动化2105班学生,彭康学导团志愿者部成员,负责本讲义的审稿任务。本人才疏学浅,时间紧迫,如有任何不当之处,欢迎大家斧正。也欢迎大家找我咨询学业问题(课业、转专业等),QQ:2473762841。

\vspace{2em}

在此,对以上牺牲个人宝贵时间来完成这份讲义的同学表示衷心感谢!